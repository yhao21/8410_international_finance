
\documentclass[12pt]{article}
\usepackage{amsmath}
\DeclareMathOperator*{\argmin}{arg\,min} % thin space, limits underneath in displays
\DeclareMathOperator*{\argmax}{arg\,max} % thin space, limits underneath in displays
\newtheorem{thm}{Theorem}
\usepackage{amssymb}
\usepackage{amsfonts}
\usepackage{mathrsfs}
\usepackage{bm}
\usepackage{indentfirst}
\setlength{\parindent}{0em}
\usepackage[margin=1in]{geometry}
\usepackage{graphicx}
\usepackage{setspace}
\doublespacing
\usepackage[flushleft]{threeparttable}
\usepackage{booktabs,caption}
\usepackage{float}
\usepackage{graphicx}

\usepackage{import}
\usepackage{xifthen}
\usepackage{pdfpages}
\usepackage{transparent}

\newcommand{\incfig}[1]{%
\def\svgwidth{\columnwidth}
\import{./figures/}{#1.pdf_tex}
}




\title{International Finance}
\author{SynFerLo}
\date{July 22, 2021}


\begin{document}
\maketitle
\newpage





\section{The seven ages}

\subsection{Seven of them}
The Paleolithic Age

The Neolithic Age

The Equestrian Age

The Classical Age

The Ocean Age

The Industrial Age

The Digital Age


\subsection{The Paleolithic Age 70,000 BCE -- 10,000 BCE}
Long-distance interactions were by migration.

\subsection{The Neolithic (new stone) Age 10,000 BCE -- 3000 BCE}
1. The fundamental breakthrough was agriculture, both crop cultivation
and animal husbandry.

2. From nomadism to sedentary life in villages.

3. Human interaction widened from the clan to the village and to
politics and {\underline {trade}} between villages. (gemstones, shells,
minerals, tools.)


\subsection{The Equestrian Age 3000 BCE -- 1000 BCE}
People label it as the Copper and Bronze age. But the author prefers
to emphasize the role of the horse over that of the minerals.

1. Domesticated horse make rapid, long-distance overland transport
and communication became possible.


\subsection{The Classical (imperial) Age 1000 BCE -- 1500 CE}
The rise and intense competition of large land-based empire.

The major empires were spurred by new religious and philosophical 
outlooks, e.g., Greco-Roman world.

The imperial age ushered in trans-Eurasian trade.


\subsection{The Ocean Age 1500 CE -- 1800 CE}
1. Imperial powers of Europe conquering and colonizing tropical regions
in Africa, the Americas, and Asia.

2. Revolutionary changes in global trade ensued, e.g., multinational
corporations, transoceanic trade, mass movement of people across the
oceans (enslavement of Africans).

3. Politics also became global in scale for the first time.


\subsection{The Industrial Age 1800 -- 2000}

Technology breakthrough, e.g., steam engine, internal combustion engine.

Global populations soared (increase rapidly) as a result of massive
increases in food production.

Global hegemon occur, Great Britain, and later, the United States.


\subsection{The Digital Age 2000 -- now}
Transfer from hegemonic power to a multipolar world.


\section{The acceleration of change}
Three dimensions of long-term change: total population, the rate of
urbanization, and the global output per person.


\section{Economic scale and the pace of change}

The most fundamental reason for the takeoff of economic growth around
1800 is therefore scale. With larger market, there can be more
specialization in job tasks.

In the Equestrian Age, for the first time, humanity was organized into
recognizable states, no longer merely interspersed villages.

With the voyages of Christopher Columbus, and Vasco da Gama, and the
transition to the {\underline {Ocean Age}}, scale increased yet again.
The world population soared again as food varieties were exchanged
across the oceans. Wheat from Old World to the Americas, and maize
from the Americas to the Old World.



\section{Malthusian Pessimism}
He argued that following any rise in productivity, the world would 
simply end up with more poor people, but with {\underline {no long-
term}} solution to poverty.


\section{Transformation to Urban Life}
Economic activities are usefully categorized into {\underline {three}}
productive sectors, called the {\underline {primary}},
{\underline {secondary, and tertiary sectors.}}

{\textbf {The primary sector}} includes the production of food and
feed crops, animal products, other agriculture, and mining products.

{\textbf {The secondary sector}}, or industrial sector, involves 
the transformation of primary commodities to final products, e.g., 
building, machinery, processed foods, and electric power.

{\textbf {The tertiary sector}} involves services that support 
productive activities, individual wellbeing, and governance.



\section{The interplay of geography, technology, and institutions}
Economists have long debated whether economic wellbeing and progress
are the results of geography, technology, or institutions.
Clearly, this debate is {\underline {misguided.}} These three domains
are interdependent; we cannot understand economic history and economic
change without taking all three into account.
























\end{document}

